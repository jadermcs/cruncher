\documentclass{article}
\usepackage[a4paper, total={6in, 10in}]{geometry}
\usepackage[T1]{fontenc}
\usepackage{import}
\usepackage{xifthen}
\usepackage{pdfpages}
\usepackage{transparent}

\newcommand{\incfig}[1]{%
    \def\svgwidth{\columnwidth}
    \import{./figures/}{#1.pdf_tex}
}

\title{Cruncher\\
\large A Semantically Consistent Language for Handling Files}
\author{Jader M. C. de Sá}
\date{\today}

\begin{document}

\maketitle

\section{Introduction}

i) área da Ciência da Computação a que a linguagem se destina;
This language is designed for data processing, specifically files and folder processing.



ii) descrição da linguagem formal para a qual será implementado o tradutor (ou seja, a gramática de tal linguagem), contendo também descrição informal dos aspectos inerentes à linguagem e a gramática subjacente;



iii) motivação para a escolha da linguagem/área: que tipos de problemas resolve; porque é adequada para descrição/apresentação de soluções na área da Ciência da Computação a que se destina; quais as facilidades que garante ao programador/usuário leigo; quais as dificuldades relativas à implementação da linguagem com as quais você irá lidar no seu trabalho;
It is a semantically consistent language in the sense that many distinct usual shell
operations will follow a intuitive syntax


iv) descrição breve da semântica da linguagem.

\begin{figure}
    \centering
    \incfig{drawing}
    \caption{Caption}
    \label{fig:my_label}
\end{figure}

\begin{table}[]
\centering
\caption{Operations description}
\label{tab:onmetrics}
\begin{tabular}{|r|c|l|}
\hline
Operation   & Symbol & Description\\ \hline
Move        & >      & Move given files or directories \\
Rename      & \&     & Renames given files or directories \\
Compression & @      & Compress files and/or directories using tar \\
Hash        & \#     & Calculates the hash \\
 &  &  \\
 &  &  \\
\hline
\end{tabular}
\end{table}

\end{document}
