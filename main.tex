\documentclass{article}
\usepackage[a4paper, total={6in, 10in}]{geometry}
\usepackage[T1]{fontenc}

\newcommand{\incfig}[1]{%
    \def\svgwidth{\columnwidth}
    \import{./figures/}{#1.pdf_tex}
}

\title{Cruncher\\
\large A Semantically Consistent Language for Handling Files}
\author{Jader M. C. de Sá}
\date{\today}

\begin{document}

\maketitle

\section{Introduction}

% motivação para a escolha da linguagem/área: que tipos de problemas resolve; porque é adequada para descrição/apresentação de soluções na área da Ciência da Computação a que se destina; quais as facilidades que garante ao programador/usuário leigo; quais as dificuldades relativas à implementação da linguagem com as quais você irá lidar no seu trabalho;


The Unix Shell is the general denomination for a command line interpreter
(and a scripting language) that provides for users and systems admins an
interface to control the execution of the system \cite{negus2010linux},
this names comes from earlier times when operating systems had this only
interface covering the operating system. It provides a way to create
executable scripts, execute programs, manage filesystems, compile code and
other general computer manipulation tasks \cite{negus2010linux,blum2008linux}.

Although shell be a very old computer technology later being replaced by
graphical interfaces, many users (specially super-users) prefer this less
intuitive interface \cite{negus2010linux,newham2005learning}. Many shells were
developed with similar behaviour but distinct orientation, to cite a few, the
most popular ones are Bash \cite{bash}, Zsh \cite{zsh} and Fish \cite{fish}.

External tools were developed to supply the needs of Unix users
but not integrated in the any shell scripting language so they presents an
inconsistent language for files handling and operations, like tar, sed, md5
although all of those have a common paradigm.

In modern \textbf{data processing}, a computing paradigm emmerged dominating
as the most adopted by major frameworks, like HadoopMR, Spark and Kafka.
MapReduce is a computational paradigm that decompose computations in a
Map operation and a following reduction has a clear and efficient model of
computation besides not being so clear how to describe any computation in
this paradigm \cite{afrati2012vision}.

This paper proposes a language designed for \textbf{data processing},
specifically files and folder processing in a standardized syntax, it is
semantically consistent language in the sense that many distinct usual shell
operations will follow a intuitive syntax.

\section{Formal Description}

% descrição da linguagem formal para a qual será implementado o tradutor (ou seja, a gramática de tal linguagem), contendo também descrição informal dos aspectos inerentes à linguagem e a gramática subjacente;
In this section the formal grammar is presented.
\\

PROCEDURE $\rightarrow$ SELECTION \$OPERATION AGGREGATION DESTINY
\\

SELECTION $\rightarrow$ LIKE|PATH|MAPPING in SELECTION if EXPRESSION
\\

OPERATION $\rightarrow$ >|\&|!|@|\# (PARAM-LIST)
\\

AGGREGATION $\rightarrow$ .|\_
\\

DESTINY $\rightarrow$ PATH
\\

PATH $\rightarrow$ type:string
\\

MAPPING $\rightarrow$ ff|rr|cc|++ PARAM-LIST
\\

EXPRESSION $\rightarrow$ LIKE
\\

PARAM-LIST $\rightarrow$ PARAM-LIST PARAM|PARAM
\\

LIKE $\rightarrow$ type:string

\section{Semantics}
The language is inspired in the MapReduce paradigm, which is dominating the
frameworks design for data processing. The main statement in the language
is the PROCEDURE, it summarizes the general MapReduce operation in an
intuitive syntax as the \$\_ operator is succinct and mnemonical.

The SELECTION acts in the folder dimension and the EXPRESSION in the
file dimension.

If no AGGREGATION is applied the DESTINY must have the same size that
SELECTION.

%The LIKE


\begin{table}[ht]
\centering
\caption{Operations description}
\label{tab:operations}
\begin{tabular}{|r|c|l|}
\hline
Operation   & Symbol & Description\\ \hline
Move        & >      & Move given files or directories \\
Rename      & \&     & Renames given files or directories \\
Edit        & !      & Stream editor operations \\
Compression & @      & Compress files and/or directories using tar \\
Hash        & \#     & Calculates the hash \\
\hline
\end{tabular}
\end{table}

\begin{table}[ht]
\centering
\caption{Maps description}
\label{tab:maps}
\begin{tabular}{|r|c|l|}
\hline
Mapping   & Symbol & Description\\ \hline
Identity  &        & Return same given value \\
Filter    & ff     & Remove a given pattern \\
Replace   & rr     & Replace a given pattern \\
Cut       & cc     & Remove a substring \\
Concat    & ++     & Replace a given pattern \\
\hline
\end{tabular}
\end{table}

\bibliographystyle{unsrt}
\bibliography{ref.bib}

\end{document}
