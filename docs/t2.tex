\documentclass{article}
\usepackage[a4paper, total={6in, 10in}]{geometry}
\usepackage[T1]{fontenc}
\usepackage{url}
\usepackage[portuguese]{babel}

\title{Cruncher\\
\large A Semantically Consistent Language for Handling Files}
\author{Jader M. C. de S\'{a}}
\date{\today}

\begin{document}

\maketitle

\section{Introdu\c{c}\~{a}o}

A sintaxe nessa implementação foi baseada na refencia de 2010 da linguagem
haskell \cite{marlow2010haskell}

Although shell be a very old computer technology later being replaced by
graphical interfaces, many users (specially super-users) prefer this less
intuitive interface \cite{negus2010linux,newham2005learning}. Many shells were
developed with similar behaviour but distinct orientation, to cite a few, the
most popular ones are Bash \cite{bash}, Zsh \cite{zsh} and Fish \cite{fish}.

External tools were developed to supply the needs of Unix users
but not integrated in the any shell scripting language so they presents an
inconsistent language for files handling and operations, like tar, sed, md5
although all of those have a common paradigm.

In modern \textbf{data processing}, a computing paradigm emmerged dominating
as the most adopted by major frameworks like HadoopMR, Spark and Kafka.
MapReduce is a computational paradigm that decompose computations in a
Maps and a following reduction, it has a clear and efficient model of
computation besides not being so clear how to describe any computation in
this paradigm \cite{afrati2012vision}.

This paper proposes a language designed for \textbf{data processing},
specifically files and folder processing in a standardized syntax, it is
semantically consistent language in the sense that many distinct usual shell
operations will follow a intuitive syntax.

A main difficult in developing an interpreter to this is the type
check for the DESTINY statement (presented later).


\section{Semantics}
The main statement in the language is the PROCEDURE, it summarizes the
general MapReduce operation in an intuitive syntax as the \$\_ operator is
succinct and mnemonical, the SELECTION statement is used to describe the
files or folders that the mapping OPERATION will be applied, the
AGGREGATION specifies which gatter method to used, summarize a result in
a single file or apply each operation in separated files. The DESTINY
statement specifies the output file/directory to send the mapping results.

The SELECTION could be described in three ways, a regular expression
to pattern match the many paths, a direct path and a list comprehension of
paths, the list comprehension of paths contain a control statement named
EXPRESSION for filtering based on a regex match, it only acts in the files
dimension.

OPERATION statement is a succinct caller for external tools, like sed, tar
sha256sum, and others. Then the result is send to DESTINY, if no AGGREGATION
is applied the DESTINY must have the same size that SELECTION.

%The LIKE


\begin{table}[ht]
\centering
\caption{Operations description}
\label{tab:operations}
\begin{tabular}{|r|c|l|}
\hline
Operation   & Symbol & Description\\ \hline
Move        & >      & Move given files or directories \\
Rename      & \&     & Renames given files or directories \\
Edit        & !      & Stream editor operations \\
Compression & @      & Compress files and/or directories using tar \\
Hash        & \#     & Calculates the hash \\
\hline
\end{tabular}
\end{table}

\begin{table}[ht]
\centering
\caption{Maps description}
\label{tab:maps}
\begin{tabular}{|r|c|l|}
\hline
Mapping   & Symbol & Description\\ \hline
Identity  & ..     & Return same given value \\
Filter    & ff     & Remove a given pattern \\
Replace   & rr     & Replace a given pattern \\
Cut       & cc     & Remove a substring \\
Concat    & ++     & Replace a given pattern \\
\hline
\end{tabular}
\end{table}

\bibliographystyle{unsrt}
\bibliography{ref.bib}

\end{document}
